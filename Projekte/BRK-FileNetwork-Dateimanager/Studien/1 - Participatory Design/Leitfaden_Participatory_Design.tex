\input{mmd-ba-notes-header}
\def\mytitle{Leitfaden für Participtory Design}
\def\myauthor{Anastasia Kazakova, Bengt Lüers}
\def\latexmode{memoir}
\def\bibliocommand{\bibliography{Literaturliste}}
\def\bibliostyle{chicago}
\input{mmd-ba-notes-begin-doc}
\pagebreak

\chapter{Leitfaden}
\label{leitfaden}

\section{Aufbau 1}
\label{aufbau1}

\subsection{Materialien}
\label{materialien}

\begin{enumerate}
\item DIN A3 - Bogen, weiß

\item Bunte Stifte

\item Dateikärtchen

\end{enumerate}

\subsection{Aufgaben}
\label{aufgaben}

\begin{enumerate}
\item Sortiere die zusammengehörigen Dateien so, dass diese Zusammengehörigkeit sieht. Erkläre wie diese Sortierung zu verstehen ist.

\item Nehme zwei Gruppen - \emph{Bachelorarbeit}, \emph{Experiment}. Die Datei \emph{Experimentrohdaten.xls} gehört zu den beiden Gruppen. Zeige die Zugehörigkeit. Erläutere deine Überlegungen.

\item Die Dateien \emph{Uni-Logo.svg} und \emph{Uni-Logo.png} gehören zu den folgenden Gruppen: \emph{Bachelorarbeit}, \emph{Experiment}, \emph{Bilder}. Wie würdest du diese Zugehörigkeit abbilden. Erkläre.

\item Schau dir die Musikdateien an. Sie stellen eine Gruppe und zwei Untergruppen dar. Bilde es ab. Erläutere.

\item Du willst mit den vier Dateien aus vier unterschiedlichen Gruppen arbeiten. Wie würdest du hier darstellen, dass sie im Vordergrund sind? Wie würdest du deren Zugehörigkeit zu den jeweiligen Gruppen darstellen.

\item Diese drei Gruppen von Dateien gehören zu den unterschiedlichen Zeiten. Erste Gruppe stellt die neuen Dateien dar. Die Zweite die Dateien mit denen du vor 2 Tagen gearbeitet hat. Die Dateien der dritten Gruppe gehören zu den Archivdateien. Bilde es ab.

\end{enumerate}

\pagebreak

\section{Aufbau 2}
\label{aufbau2}

\subsection{Materialien}
\label{materialien}

\begin{enumerate}
\item DIN A3 - Bogen, weiß

\item Bunte Stifte

\item Dateikärtchen

\item Verbindungen

\item Kreise

\item Tags

\end{enumerate}

\subsection{Aufgaben}
\label{aufgaben}

\begin{enumerate}
\item Sortiere die zusammengehörigen Dateien so, dass diese Zusammengehörigkeit sieht. Erkläre wie diese Sortierung zu verstehen ist.

\item Nehme zwei Gruppen - \emph{Bachelorarbeit}, \emph{Experiment}. Die Datei \emph{Experimentrohdaten.xls} gehört zu den beiden Gruppen. Zeige die Zugehörigkeit. Erläutere deine Überlegungen.

\item Die Dateien \emph{Uni-Logo.svg} und \emph{Uni-Logo.png} gehören zu den folgenden Gruppen: \emph{Bachelorarbeit}, \emph{Experiment}, \emph{Bilder}. Wie würdest du diese Zugehörigkeit abbilden. Erkläre.

\item Schau dir die Musikdateien an. Sie stellen eine Gruppe und zwei Untergruppen dar. Bilde es ab. Erläutere.

\item Du willst mit den vier Dateien aus vier unterschiedlichen Gruppen arbeiten. Wie würdest du hier darstellen, dass sie im Vordergrund sind? Wie würdest du deren Zugehörigkeit zu den jeweiligen Gruppen darstellen.

\item Diese drei Gruppen von Dateien gehören zu den unterschiedlichen Zeiten. Erste Gruppe stellt die neuen Dateien dar. Die Zweite die Dateien mit denen du vor 2 Tagen gearbeitet hat. Die Dateien der dritten Gruppe gehören zu den Archivdateien. Bilde es ab.

\end{enumerate}

\input{mmd-ba-notes-footer}

\end{document}
