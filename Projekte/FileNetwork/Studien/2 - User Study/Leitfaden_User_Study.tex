\documentclass[12pt,]{article}
\usepackage[T1]{fontenc}
\usepackage{ngerman}
\usepackage{lmodern}
\usepackage{amssymb,amsmath}
\usepackage{ifxetex,ifluatex}
\usepackage{fixltx2e} % provides \textsubscript
% use microtype if available
\IfFileExists{microtype.sty}{\usepackage{microtype}}{}
\ifnum 0\ifxetex 1\fi\ifluatex 1\fi=0 % if pdftex
  \usepackage[utf8]{inputenc}

\else % if luatex or xelatex
  \usepackage{fontspec}
  \ifxetex
    \usepackage{xltxtra,xunicode}
  \fi
  \defaultfontfeatures{Mapping=tex-text,Scale=MatchLowercase}
  \newcommand{\euro}{€}
    \setmainfont{Georgia}
    \setsansfont{Arial}
    \setmonofont{Bitstream Vera Sans Mono}
\fi

\usepackage{fancyhdr}
\pagestyle{fancy}
\pagenumbering{arabic}
%\lhead{\itshape File Network: Aufgaben für Userstudie}
%\chead{}
%\rhead{\itshape{\nouppercase{\leftmark}}}

%\cfoot{}
%\rfoot{\thepage}

\ifxetex
  \usepackage[setpagesize=false, % page size defined by xetex
              unicode=false, % unicode breaks when used with xetex
              xetex]{hyperref}
\else
  \usepackage[unicode=true]{hyperref}
\fi
\hypersetup{breaklinks=true,
            bookmarks=true,
            pdfauthor={},
            pdftitle={File Network: Aufgaben für Userstudie},
            colorlinks=true,
            urlcolor=blue,
            linkcolor=magenta,
            pdfborder={0 0 0}}
\setlength{\parindent}{0pt}
\setlength{\parskip}{6pt plus 2pt minus 1pt}
\setlength{\emergencystretch}{3em}  % prevent overfull lines

\title{File Network: Aufgaben für Userstudie}
\author{}
\date{}

\begin{document}
\maketitle

\newpage

\section{Szenario: Herunterladen}

Sie haben sich ein neues System aufgesetzt und brauchen zum Arbeiten
einige Dateien.

\subsection{Aufgabe 1}

Laden die benötigten Dateien herunter:

\begin{itemize}
\item
  \texttt{ExperimentAufbau.doc}
\item
  \texttt{ExperimentRohdaten.xls}
\item
  \texttt{ExperimentAuswertung.doc}
\item
  \texttt{ExperimentPräsentation.pdf}
\end{itemize}

\newpage

\section{Szenario: Einsortieren}

Sie benutzen die Übersicht, um Dateien einzusortieren.

\subsection{Aufgabe 2}

Bitte sortieren Sie die Dateien folgender Maßen in der angegebenen
Reihenfolge ein.

\begin{itemize}
\item
  \texttt{ExperimentAufbau.doc} in einen neuen Kreis
\item
  \texttt{ExperimentRohdaten.xls} in den bestehenden Kreis
\item
  \texttt{ExperimentPräsentation.pdf} in einen neuen Kreis
\item
  \texttt{ExperimentAuswertung.doc} in den ersten Kreis
\item
  \texttt{ExperimentAuswertung.doc} in den zweiten Kreis
\end{itemize}

\newpage

\subsection{Aufgabe 3}

Legen Sie die Farbe vom größeren Kreis fest.

\newpage

\subsection{Aufgabe 4}

Entfernen Sie die Datei \texttt{ExperimentAuswertung.doc} aus dem
größeren Kreis.

\newpage

\section{Szenario: Verbindungen}

Sie haben das System jetzt eine Weile benutzt und einige weitere Dateien
eingeordnet.

\subsection{Aufgabe 5}

Sehen Sie sich die Vorschläge für neue Verbindungen an.

\newpage

\subsection{Aufgabe 6}

Bestätigen Sie die vorgeschlagene Verbindung zwischen
\texttt{ExperimentAuswertung.doc} und \texttt{ExperimentRohdaten.xls}.

\newpage

\subsection{Aufgabe 7}

Löschen Sie die Verbindung zwischen den Dateien
\texttt{ExperimentAuswertung.doc} und \texttt{ExperimentRohdaten.xls}.

\newpage

\subsection{Aufgabe 8}

Navigieren Sie zu der Datei \texttt{ExperimentPräsentation.pdf}.

\newpage

\subsection{Aufgabe 9}

Navigieren Sie zu der Datei \texttt{ExperimentRohdaten.xls}.

\newpage

\section{Szenario: Aufräumen}

Sie haben einige Dateien geöffnet von denen nur einige einsortiert sind.

\subsection{Aufgabe 10}

Räumen Sie die Arbeitsfläche auf.

\end{document}
