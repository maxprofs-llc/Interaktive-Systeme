%
%	Configure LaTeX to produce a PDF presentation using the beamer class
%

\documentclass[ignorenonframetext,11pt]{beamer}
\usepackage[ngerman]{babel}
\usepackage[utf8]{inputenc}		% For UTF-8 support
\usepackage[T1]{fontenc}

\usepackage{graphicx}


\usepackage{beamerthemesplit}
\usepackage{patchcmd}
\usepackage{tabulary}		% Support longer table cells
\usepackage{booktabs}		% Support better tables
\usepackage[sort&compress]{natbib}

\usepackage{framed}			% Allow background color for images
\definecolor{shadecolor}{named}{white}


\usepackage{subfigure}

\let\oldSubtitle\subtitle


% Configure default metadata
\input{mmd-default-metadata}


%\AtBeginSection[]
%{
 %\begin{frame}
    %\frametitle{Inhalt}
	%\tableofcontents %[currentsection,currentsubsection]
   %\end{frame}
%}


\long\def\citefoot#1{\let\thefootnote\relax\footnotetext{\citet{#1}} }

\def\mytitle{FileNetwork: Konzept zur Requirements Analyse}
\def\myauthor{Anastasia Kazakova, Jens Rauch, Marco Brochert}
\def\mydate{5. Mai 2012}
\def\latexmode{beamer}
\def\latexxslt{beamer}
\def\theme{keynote-IntSysTheme}
\input{mmd-natbib-plain}
%
%	Get ready for the actual document
%

%
% Use default MMD metadata for beamer equivalents
%

\ifx\subtitle\undefined
\else
	\oldSubtitle{\subtitle}
\fi

\ifx\affiliation\undefined
\else
	\institute{\affiliation}
\fi

\ifx\mydate\undefined
	\def\mydate{\today}
\else
	\date{\mydate}
\fi

\ifx\event\undefined
\else
	\date[\mydate]{\mydate~ / \event }
\fi


%\input{mmd-title}

% Show "current/total" slide counter in footer
\title[\mytitle\hspace{2em}\insertframenumber/
\inserttotalframenumber]{\mytitle}


\author{\myauthor}
\addtolength{\parskip}{\baselineskip}

\ifx\theme\undefined
\else
	\usetheme{\theme}
\fi

\begin{document}
\frame[plain]{\setlength\parskip{0pt}\titlepage}

\# \#

\section{State of the Art}
\label{stateoftheart}

\begin{frame}

\frametitle{OS gebotene Möglichkeiten}
\label{osgebotenemglichkeiten}

\textbf{Einsortieren}

\begin{itemize}
\item manuell

\item taggen

\end{itemize}

\textbf{Suche}

\begin{itemize}
\item in den Ordnern

\item glibale Suche

\item nach Tags

\end{itemize}

\end{frame}

\begin{frame}

\frametitle{Zusätzliche Anwendungen}
\label{zustzlicheanwendungen}

\begin{itemize}
\item Unterstützung durch bessere und bequeme GUI

\end{itemize}

\textbf{Einsortieren}

\begin{itemize}
\item Taggen beim Herunterladen oder Erstellen der Datein

\end{itemize}

\textbf{Suchen}

\begin{itemize}
\item Nach Tags

\end{itemize}

\end{frame}

\begin{frame}

\frametitle{Aus der Literatur}
\label{ausderliteratur}

\textbf{Suchen}\footnote{S. Dekeyser, R. Watson, L. Motrøen \emph{A Model, Schema, and Interface for Metadata File Systems}.}

\begin{itemize}
\item nach Metadaten (z.B. Google Desktop)

\item nach Inhalt (z.B. Google Desktop)

\end{itemize}

\textbf{VennFS2-Tool}\footnote{R. De Chiara, U. Erra, V. Scarano \emph{A Visual Adaptive Interface to File Systems}.}

\begin{itemize}
\item eine Datei gehört zu mehreren Kategorien

\item Vorschläge der Anordnung

\item Visualisierung durch Venn-Diagramme

\end{itemize}

\end{frame}

\begin{frame}

\frametitle{Mögliche Konzepte}
\label{mglichekonzepte}

\begin{itemize}
\item automatische Assoziation von Dateien auf Basis der Gebrauchshistorie

\item automatische\slash manuelle Zuordnung von (optional hierarchisch geordneten) Tags

\item interaktive Visualisierung des Filenetworks über Filter und Verfolgen von Assoziationen zwischen Dateien\slash Tags

\end{itemize}

\end{frame}

\section{Anforderungserhebung}
\label{anforderungserhebung}

\begin{frame}

\frametitle{Zielgruppe und Methode}
\label{zielgruppeundmethode}

\textbf{Zielgruppe}

\begin{itemize}
\item regelmäßige PC-Nutzer

\item ausschießen von Kindern und Mneschen über 60

\end{itemize}

\textbf{Methode}

\begin{itemize}
\item Qualitatives Leitfadeninterview

\item Beobachtung

\end{itemize}

\end{frame}

\begin{frame}

\frametitle{Leitfaden}
\label{leitfaden}

\begin{enumerate}
\item Zeigen Sie mir bitte, wie Sie Dateien, die Sie erstellen oder abspeichern, auf Ihrer
Festplatte organisieren. 

\item Notieren:

\begin{itemize}
\item OS

\item File Manager (Explorer, Nautilus {\ldots} )

\item evtl. Erweiterungen (falls es die Erweiterungen, für die Verwaltung von Musik- od. Bilddateien gibt, sich zeigen lassen, fragen wie funktioniert, ob und womit man (un-)zufrieden ist)

\end{itemize}

\item Wie gehen Sie vor, wenn Sie eine Datei suchen, deren Ablageort Sie nicht mehr wissen?

\item Wie könnten Sie gesuchte Dateien schneller auffinden?

\item Was hindert Sie daran Dateien schneller aufzufinden?

\end{enumerate}

\end{frame}

\section{Danke}
\label{danke}

\textbf{Fragen? Anmerkungen?}\mode<all>


%
%	MultiMarkdown beamer class footer file
%

% Back Matter
\if@mainmatter
\backmatter
\fi

\ifx\bibliocommand\undefined
\else
	\part{Literatur}
	\begin{frame}[allowframebreaks]
	\frametitle{Quellen}
	\bibliographystyle{\bibliostyle}
	\def\newblock{}
	\bibliocommand
	\end{frame}
\fi


\end{document}\mode*

