\input{mmd-beamer-IntSysTheme-header}
\def\mytitle{File Network: Prototyp und Planung des Usabilitytests}
\def\myauthor{Anastasia Kazakova, Bengt Lüers}
\def\latexmode{beamer}
\def\latexxslt{beamer}
\def\theme{keynote-IntSysTheme}
\input{mmd-natbib-plain}
\def\bibliostyle{chicago}
\input{mmd-beamer-begin-IntSysTheme-doc}
\# \#

\section{Prototyp}
\label{prototyp}

\begin{frame}

\frametitle{Verbessertes Prototyp}
\label{verbessertesprototyp}

\textbf{Demo}

\end{frame}

\section{Studie}
\label{studie}

\begin{frame}

\frametitle{Vorläufiger Plan}
\label{vorlufigerplan}

\begin{itemize}
\item Tests anhand der Aufgaben

\item Beobachten

\item Interview

\end{itemize}

\end{frame}

\begin{frame}

\frametitle{Aufgaben: Beispiel}
\label{aufgaben:beispiel}

\textbf{Szenario 1: Herunterladen}

\emph{Sie haben sich ein neues System aufgesetzt und brauchen zum Arbeiten einige Dateien.}

\textbf{Aufgabe 1:} Laden die folgenden benötigten Dateien herunter:

\begin{itemize}
\item \texttt{Experimentaufbau.doc}

\item \texttt{Experimentrohdaten.xls}

\item \texttt{Experimentauswertung.doc}

\item \texttt{Experimentpräsentation.pdf}

\end{itemize}

\end{frame}

\begin{frame}

\frametitle{Beobachtung}
\label{beobachtung}

\begin{itemize}
\item Think aloud

\item Screencast

\item Notieren der Probleme

\item Ob die Aufgabe gelöst wurde oder nicht

\end{itemize}

\end{frame}

\begin{frame}

\frametitle{Interview}
\label{interview}

\begin{itemize}
\item Ob das Kozept gefallen hat

\item Ob man mit dem FileNetwork klargekommen ist

\item Verbesserungsvorschläge

\end{itemize}

\end{frame}

\begin{frame}

\frametitle{``Tutorium'' vor der Studie}
\label{tutoriumvorderstudie}

\begin{itemize}
\item Ganz neues System

\item Elemente erläutern

\item Bedienung erklären

\item Üben lassen

\end{itemize}

\end{frame}

\begin{frame}

\frametitle{Vielen Dank für die Aufmerksamkeit!}
\label{vielendankfrdieaufmerksamkeit}

\textbf{Fragen?}

\end{frame}

\mode<all>
%
%	MultiMarkdown beamer class footer file
%

% Back Matter
\if@mainmatter
\backmatter
\fi

\ifx\bibliocommand\undefined
\else
	\part{Literatur}
	\begin{frame}[allowframebreaks]
	\frametitle{Quellen}
	\bibliographystyle{\bibliostyle}
	\def\newblock{}
	\bibliocommand
	\end{frame}
\fi


\end{document}\mode*

