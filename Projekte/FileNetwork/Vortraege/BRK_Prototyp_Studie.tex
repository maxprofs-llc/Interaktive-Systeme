%
%	Configure LaTeX to produce a PDF presentation using the beamer class
%

\documentclass[ignorenonframetext,11pt]{beamer}
\usepackage[ngerman]{babel}
\usepackage[utf8]{inputenc}		% For UTF-8 support
\usepackage[T1]{fontenc}

\usepackage{graphicx}


\usepackage{beamerthemesplit}
\usepackage{patchcmd}
\usepackage{tabulary}		% Support longer table cells
\usepackage{booktabs}		% Support better tables
\usepackage[sort&compress]{natbib}

\usepackage{framed}			% Allow background color for images
\definecolor{shadecolor}{named}{white}


\usepackage{subfigure}

\let\oldSubtitle\subtitle


% Configure default metadata
\input{mmd-default-metadata}


%\AtBeginSection[]
%{
 %\begin{frame}
    %\frametitle{Inhalt}
	%\tableofcontents %[currentsection,currentsubsection]
   %\end{frame}
%}


\long\def\citefoot#1{\let\thefootnote\relax\footnotetext{\citet{#1}} }

\def\mytitle{File Network: Prototyp und Planung des Usabilitytests}
\def\myauthor{Anastasia Kazakova, Bengt Lüers}
\def\latexmode{beamer}
\def\latexxslt{beamer}
\def\theme{keynote-IntSysTheme}
\input{mmd-natbib-plain}
\def\bibliostyle{chicago}
%
%	Get ready for the actual document
%

%
% Use default MMD metadata for beamer equivalents
%

\ifx\subtitle\undefined
\else
	\oldSubtitle{\subtitle}
\fi

\ifx\affiliation\undefined
\else
	\institute{\affiliation}
\fi

\ifx\mydate\undefined
	\def\mydate{\today}
\else
	\date{\mydate}
\fi

\ifx\event\undefined
\else
	\date[\mydate]{\mydate~ / \event }
\fi


%\input{mmd-title}

% Show "current/total" slide counter in footer
\title[\mytitle\hspace{2em}\insertframenumber/
\inserttotalframenumber]{\mytitle}


\author{\myauthor}
\addtolength{\parskip}{\baselineskip}

\ifx\theme\undefined
\else
	\usetheme{\theme}
\fi

\begin{document}
\frame[plain]{\setlength\parskip{0pt}\titlepage}

\# \#

\section{Prototyp}
\label{prototyp}

\begin{frame}

\frametitle{Verbessertes Prototyp}
\label{verbessertesprototyp}

\textbf{Demo}

\end{frame}

\section{Studie}
\label{studie}

\begin{frame}

\frametitle{Vorläufiger Plan}
\label{vorlufigerplan}

\begin{itemize}
\item Tests anhand der Aufgaben

\item Beobachten

\item Interview

\end{itemize}

\end{frame}

\begin{frame}

\frametitle{Aufgaben: Beispiel}
\label{aufgaben:beispiel}

\textbf{Szenario 1: Herunterladen}

\emph{Sie haben sich ein neues System aufgesetzt und brauchen zum Arbeiten einige Dateien.}

\textbf{Aufgabe 1:} Laden die folgenden benötigten Dateien herunter:

\begin{itemize}
\item \texttt{Experimentaufbau.doc}

\item \texttt{Experimentrohdaten.xls}

\item \texttt{Experimentauswertung.doc}

\item \texttt{Experimentpräsentation.pdf}

\end{itemize}

\end{frame}

\begin{frame}

\frametitle{Beobachtung}
\label{beobachtung}

\begin{itemize}
\item Think aloud

\item Screencast

\item Notieren der Probleme

\item Ob die Aufgabe gelöst wurde oder nicht

\end{itemize}

\end{frame}

\begin{frame}

\frametitle{Interview}
\label{interview}

\begin{itemize}
\item Ob das Kozept gefallen hat

\item Ob man mit dem FileNetwork klargekommen ist

\item Verbesserungsvorschläge

\end{itemize}

\end{frame}

\begin{frame}

\frametitle{``Tutorium'' vor der Studie}
\label{tutoriumvorderstudie}

\begin{itemize}
\item Ganz neues System

\item Elemente erläutern

\item Bedienung erklären

\item Üben lassen

\end{itemize}

\end{frame}

\begin{frame}

\frametitle{Vielen Dank für die Aufmerksamkeit!}
\label{vielendankfrdieaufmerksamkeit}

\textbf{Fragen?}

\end{frame}

\mode<all>
%
%	MultiMarkdown beamer class footer file
%

% Back Matter
\if@mainmatter
\backmatter
\fi

\ifx\bibliocommand\undefined
\else
	\part{Literatur}
	\begin{frame}[allowframebreaks]
	\frametitle{Quellen}
	\bibliographystyle{\bibliostyle}
	\def\newblock{}
	\bibliocommand
	\end{frame}
\fi


\end{document}\mode*

