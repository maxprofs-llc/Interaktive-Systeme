\section{Die geschichtliche Entwicklung des Feldes HCI}

$\rightarrow$
\url{http://www.slideshare.net/mrettig/interaction-design-history}

Grober Verlauf:

\begin{itemize}
\item
  Use the Machine
\item
  Use the Software
\item
  Perform a Task
\item
  Experience
\item
  Connect
\item
  Dynamically Enable
\end{itemize}

\section{Beispiel: OLB Baunfinanzierung}

\subsection{Ansätze}

\begin{enumerate}[1.]
\item
  Interview mit einem Stakeholder
\item
  Paper-/Wireframe-Prototyping
\item
  Open Card Sorting
\end{enumerate}

\subsection{Iterative Design Phase}

\begin{enumerate}[1.]
\item
  Cognitive Walkthrough
\item
  Usability-Test
\end{enumerate}

\section{User Requirements}

Was macht Projekte erfolgreich?

\begin{itemize}
\item
  User Involvement
\item
  Clear Statement of Requirements
\end{itemize}

\begin{figure}[htbp]
\centering
\includegraphics{inc/HCDP.pdf}
\caption{image}
\end{figure}

\subsection{Context of use analysis}

\begin{itemize}
\item
  Wer wird das System benutzen?
\item
  Wer hat außerdem Interesse daran, dass alles läuft?
\item
  Welche Charakteristiken haben diese Gruppen?
\item
  Wie werden Vorgänge normalerweise ausgeführt?
\item
  Umgebungen:

  \begin{itemize}
  \item
    technisch: Hardware, Software
  \item
    physikalisch: Wetter, Beleuchtung, \ldots{}
  \item
    sozial: Arbeitsweisen, Organisationsstrukturen, Einstellungen
  \end{itemize}
\end{itemize}

\subsection{Umfragen}

\subsubsection{Fragebögen}

\begin{itemize}
\item
  Oft für statistischen Nutzen
\item
  Antwortmöglichkeiten:

  \begin{itemize}
  \item
    Ja / Nein-Boxen
  \item
    Mehrere Optionen
  \item
    Likert-Skala (Grad der Zustimmung)
  \item
    Offene Fragen
  \end{itemize}
\item
  Vorteile:

  \begin{itemize}
  \item
    Zeit- und Kosteneffizient
  \item
    Inhaltlich frei
  \item
    Relativ fehlerfrei wenn standardisiert
  \item
    Einfach zu verwalten
  \end{itemize}
\item
  Nachteile:

  \begin{itemize}
  \item
    Ergebnis hängt stark vom Befragten ab
  \item
    Vorauswahl dadurch, dass die TN eventuell nicht repräsentativ sind
  \end{itemize}
\end{itemize}

\subsubsection{Interviews}

\begin{itemize}
\item
  Strukturiert (weniger Kontextinformationen, einfacher zu
  interpretieren)
\item
  Semi-strukturiert
\item
  Offen (abhängig vom Können des Interviewers)
\item
  Vorteile:

  \begin{itemize}
  \item
    Einfach, effizient und praktisch
  \item
    Hohe Validität
  \item
    Nachfragen möglich
  \item
    Einfach aufzunehmen
  \end{itemize}
\item
  Nachteile:

  \begin{itemize}
  \item
    Abhängig vom Können des Interviewers
  \item
    Interviewer könnte Antworten beeinflussen
  \item
    Zeitaufwändig und teuer
  \item
    Nicht verlässlich
  \item
    Ergebnisse sind schwierig zu verallgemeinern
  \end{itemize}
\end{itemize}

\subsubsection{Zielgruppen}

\begin{itemize}
\item
  6--12 Teilnehmer
\item
  Konzentration auf ein Thema, $\rightarrow$ Gruppendiskussion
\item
  Ideen erzeugen, Produkte vergleichen, \ldots{}
\item
  Heterogenität ist nützlich, aber nicht über Hierarchien oder in
  gengnesätzlichen Ansichten
\item
  Vorbereitung:

  \begin{itemize}
  \item
    Zeit einplanen (1--3 Stunden)
  \item
    Fragen vorbereiten (4--10)
  \item
    TN einladen und Ziele erklären
  \item
    Material bereitstellen
  \end{itemize}
\item
  Vorteile:

  \begin{itemize}
  \item
    Breit gestreute und qualitative Informationen
  \item
    Zeigt Konfliktpotenzial auf
  \item
    Günstig und einfach
  \end{itemize}
\item
  Nachteile:

  \begin{itemize}
  \item
    Teilnehmer sind nicht repräsentativ
  \item
    Rolle des Moderators ist groß
  \item
    Einzelne TN können dominieren
  \item
    Nicht quantitativ
  \item
    Schwer zu verallgemeinern
  \end{itemize}
\end{itemize}

\subsection{Ethnographische Studien}

\begin{itemize}
\item
  Beobachtung von Menschen im Supermarkt, zu Hause, bei der Arbeit,
  \ldots{}
\item
  Ziel: Verhalten verstehen
\item
  Aufzeichnungen mit Papier und Stift / Audio und Video /
  Computerlogging / Tagebuch (vom TN geschrieben)
\item
  Tagebuch:

  \begin{itemize}
  \item
    Informationen zu Ort, Zeit, was passiert ist
  \item
    Alternativ zum Schreiben: Diktiergerät, Kamera, E-Mail-Adresse,
    \ldots{}
  \item
    Hinterher intensives Interview
  \item
    Vorteile:

    \begin{itemize}
    \item
      Billig
    \item
      Über längere Dauert möglich
    \item
      Gut für Nutzungskontext
    \end{itemize}
  \item
    Nachteile:

    \begin{itemize}
    \item
      Hängt von Motivation ab
    \item
      Nicht verlässlich
    \end{itemize}
  \end{itemize}
\end{itemize}

\subsection{Task Analysis}

\begin{itemize}
\item
  Möglichkeiten für neue Produkte finden
\item
  Task-decomposition: abstraktere Aufgaben in Teilaufgaben unterteilen
\end{itemize}

\subsection{Studien durchführen}

\begin{itemize}
\item
  Informationsblatt und Einverständniserklärung sind wichtig
\item
  Guidelines:

  \begin{enumerate}[1.]
  \item
    Wünsche des Stakeholders erfassen
  \item
    Alle Stakeholder beachten
  \item
    Mehr als einen Repräsentaten jeder Stakeholder-Gruppe
  \item
    Datenerhebungstechniken kombinieren
  \item
    Unterstützung durch Prototypen oder Aufgabenbeschreibungen
  \item
    Pilotstudie durchführen
  \item
    Daten aufnehmen
  \item
    Zeitnah mit der Interpretation beginnen
  \item
    Interpretation vor der Analyse (WTF?!)
  \end{enumerate}
\end{itemize}

\subsection{Anforderungsspezifikation}

\subsubsection{Personas}

\begin{itemize}
\item
  Fiktionale Repräsentation eines typisches Nutzers
\item
  Hintergrundinformationen aus Literatur, Interviews, Beobachtungen,
  Statistiken
\item
  Repräsentativ aber nicht durchschnittlich
\end{itemize}

\subsubsection{Szenarien}

Erzählerische Beschreibung eines Anwendungsfalls, betrachtet dabei auch
den Kontext des Benutzers.

\subsubsection{Anwendungsfälle}

Aus dem Software Engineering, Interaktion mit der Funktionalität eines
Systems.

\subsection{Vorwissen}

\subsubsection{State of the Art Analysis}

Vergleich von existierenden Systemen.

\subsubsection{General Design Principles}

Beispiele:

\begin{itemize}
\item
  Shneiderman`s „Eight Golden Rules of Dialog Design``
\item
  ISO9241: Accessibility and Usability
\item
  Mayhew`s General Principles of User Interface Design
\item
  IBM`s Design Principles for tomorrow
\item
  Platform guidelines
\item
  Corporate Design guidelines
\end{itemize}

\section{UI Structure and Design}

\subsection{Einführung}

\subsubsection{Zielgruppe}

\begin{itemize}
\item
  Demographische Einschätzung (Alter, Geschlecht, Ort, Bildung, Arbeit,
  Einkommen, Hobbys, Ausstattung, \ldots{})
\item
  Einschätzung nach Erfahrung und Verhalten (Anfänger, Fortgeschritten,
  Experte, \ldots{})
\end{itemize}

\subsubsection{Ziele}

\begin{itemize}
\item
  Ziele der Anwendung (Unterhaltung, Bildung, Büro, Verwaltung,
  Kommunikation, Information, \ldots{})
\item
  Ziele der Benutzer (Wissen erlangen, einen Freund erreichen, ein
  Problem lösen, ein Dokument erstellen, \ldots{})
\end{itemize}

\subsubsection{Inhalt}

\begin{itemize}
\item
  ???
\end{itemize}

\subsection{Strukturdesign}

\subsubsection{Struktur}

\begin{itemize}
\item
  Hierarchien sind einfach
\item
  Ordnen nach Wichtigkeit, Granularität, Erwartungen, Bedürfnissen
\item
  Lieber in die Breite als in die Tiefe gehen
\item
  Maximale Tiefe: 5--6 Level
\end{itemize}

\subsubsection{Ausrichtung und Navigation}

\begin{itemize}
\item
  Benutzerfragen:

  \begin{itemize}
  \item
    Wo bin ich? $\rightarrow$ Brotkrumen-Navigation
  \item
    Was kann ich tun? $\rightarrow$ Beware the big button trap (???)
  \item
    Was passiert wenn ich dies tue?
  \item
    Wo komme ich her? / Wie komme ich zurück?
  \end{itemize}
\item
  Visuelles (Farben, Schriften, Bilder und Symbole) sollten einfach
  leicht zu merken sein
\item
  Ein Menü ist gut für Navigation und Orientierung
\item
  Weißraum: Trennt Informationen, hebt hervor
\end{itemize}

\begin{figure}[htbp]
\centering
\includegraphics{inc/whitespace.pdf}
\caption{image}
\end{figure}

\subsubsection{Card Sorting}

\begin{itemize}
\item
  Man fragt die Benutzer, wie sie Inhalt strukturieren und benennen
  würden
\item
  Dabei werden Muster (=\emph{Mentale Modelle}) gesucht
\item
  Gut geeignet für Menü-Kategorien und Navigation
\item
  Methode:

  \begin{itemize}
  \item
    Inhalt vorauswählen, auf ähnliche Granularität (Detaillevel) achten
  \item
    Ungefähr 30 Karten
  \item
    Kurze, schnell zu lesende aber aussagekräftige Begriffe
  \item
    Freie Karten um Begriffe zu ergänzen
  \end{itemize}
\item
  Durchführung:

  \begin{itemize}
  \item
    Teilnehmer sollten repräsentativ sein
  \item
    TN einzeln (15--30 TN) oder in 5 Gruppen à 3 TN
  \item
    Material: beschriftete und freie Karten, Stift, Gummibänder,
    Büroklammern, Klebstoff
  \item
    Am Anfang Einführung geben, dann beobachten
  \end{itemize}
\item
  Analyse:

  \begin{itemize}
  \item
    Muster durch Ordnung auf dem Tisch, am Whiteboard, \ldots{}
  \item
    Unterschiede deuten auf fehlendes oder falsches Verständnis hin
  \item
    Methoden: Multidimensional Scaling, Hierarchical Cluster Analysis
  \end{itemize}
\end{itemize}

\subsection{Bildschirmdesign und -layout}

\subsubsection{Gestaltgesetze}

\begin{itemize}
\item
  Köhler, Koffka, Werheimer (Berliner Schule), 1912: Gestaltpsychologie
\item
  Basiert auf Wahrnehmung, Bewegung, Gedächtnis, Denken, Lernen und
  Verhalten
\item
  Insgesamt über 100 Gesetze
\end{itemize}

\begin{enumerate}[1.]
\item
  Prägnanz: Einfache Formen \includegraphics{inc/gestalt_1.pdf}
\item
  Nähe: Beieinander liegende Objekte sind zusammengehörig
  \includegraphics{inc/gestalt_2.pdf}
\item
  Geschlossenheit: Fenster-Metapher \includegraphics{inc/gestalt_3.pdf}
\item
  Ähnlichkeit: Ähnliche Formen gehören zusammen
  \includegraphics{inc/gestalt_4.pdf}
\item
  Gute Fortsetzung: Kontinuierliche Formen gehören zusammen
  \includegraphics{inc/gestalt_5_1.pdf}
  \includegraphics{inc/gestalt_5_2.pdf}
\item
  Erfahrung: Neue Informationen werden in bekannte Strukturen
  eingeordnet \includegraphics{inc/gestalt_6.pdf}
\item
  Gemeinsame Bewegung: \includegraphics{inc/gestalt_7.pdf}
\end{enumerate}

\subsubsection{Farben}

\begin{itemize}
\item
  Farben sind nie neutral, können Emotionen hervorrufen und sind oft
  unterbewusst wahrgenommen
\item
  Einflüsse: Biologisch, Kulturell, Individuell
\item
  Benachbarte Farben beruhigen
\item
  Komplementäre Farben erzeugen Spannung
\item
  Maximal 4--5 Farben benutzen
\item
  Farben konsistent benutzen
\end{itemize}

\subsubsection{Bilder und Symbole}

\begin{itemize}
\item
  Illustration, Dekoration, Strukturierung
\item
  Bilder

  \begin{itemize}
  \item
    sparen Platz,
  \item
    sind leicht zu erkennen,
  \item
    Sprachunabhängig,
  \item
    einfach zu merken,
  \item
    unterbewusst wahrnehmbar
  \end{itemize}
\item
  Gute Bilder

  \begin{itemize}
  \item
    zeigen nur das wichtigste,
  \item
    kombinieren Bekanntes mit Neuem
  \item
    sprechen Emotionen an
  \end{itemize}
\end{itemize}

\subsubsection{Typographie}

\begin{itemize}
\item
  strukturiert und hebt hervor
\item
  beinhaltet Schriftart, Schriftschnitt, Größe, Farbe und Dekoration
\end{itemize}

\subsection{Keep in Mind}

Think from a user's perspective

\begin{itemize}
\item
  When, where and how will they use the system?
\item
  What are their characteristics?
\item
  Are they handicapped?
\item
  What do they expect?
\item
  What are they accustomed to?
\item
  What do they like?
\end{itemize}

Design for the actual users

\section{Gedächtnis und Aufmerksamkeit}

\begin{itemize}
\item
  Geteilte Aufmerksamkeit: Auf alles gleichzeitig achten (z.B.
  Autofahren)
\item
  Selektive Aufmerksamkeit: Konzentration auf einzelnes
\item
  Methoden:

  \begin{itemize}
  \item
    Eyetracking
  \item
    Saliency Maps (Aufmerksamkeitskarten)
  \end{itemize}
\end{itemize}

\section{Affordance, Constraints, Models und Metaphern}

\subsection{Affordanzen}

\begin{itemize}
\item
  Angebotscharakter: ``An affordance is a quality of an object, or an
  environment, which allows an individual to perform an action.''
\item
  Beispiel: Türen
\end{itemize}

\subsection{Mappings}

\begin{itemize}
\item
  Verbindung zwischen Userinterface und echter Welt
\item
  Gut: physikalische Analogie, kulturelle Standards
\item
  Beispiele: räumlich, wahrnehmbare Analogien (Schalter sieht genauso
  aus, wie das, was er bedient)
\end{itemize}

\subsection{Constraints}

\begin{itemize}
\item
  Einschränkungen sind das Gegenteil von Affordanzen und können diese
  Vergrößern
\item
  Ziel: Benutzungsfehler vermeiden, Information, die erinnert werden
  muss, reduzieren
\item
  Arten:

  \begin{itemize}
  \item
    Physikalisch: Schränken physische Operationen ein, z.B. durch eine
    Form
  \item
    Semantisch: Sich aus dem Kontext und dem Wissen über die Welt
    ergebene Einschränkungen
  \item
    Logisch: Das, was logisch erscheint
  \item
    Kulturell: Farben oder Schriften-abhängig
  \end{itemize}
\end{itemize}

\subsection{Konzeptuelle Modelle}

\begin{itemize}
\item
  Modelle sorgen dafür, dass nicht über jede Handlung nachgedacht werden
  muss, sondern Dinge automatisch erledigt werden können.
\end{itemize}

\subsection{Metaphern}

\begin{itemize}
\item
  Ein Bekannter Begriff wird als Analogie zu einem unbekannten
  Sachverhalt verwandt
\item
  Gefahr der Unter-/Überschätzung des Systems durch zu genaue Analogie
\item
  Reduktion auf Kernmerkmale
\end{itemize}

\section{Usability Guidelines}

\begin{itemize}
\item
  Definition: ``The extent to which a product can be used by specified
  users to achieve specified goals with effectiveness, efficiency and
  satisfaction in a specified context of use.'' {[}ISO 9241-11{]}
\item
  Unterschied: Effektivität (ein Ziel erreichen) und Effizienz (ein Ziel
  mit minimalem Aufwand erreichen)
\item
  Leaky Pipe Metaphor: Auf dem Weg zum Ziel werden Benutzer verloren
  (``Drop outs''), weil sie das Interface nicht richtig bedienen
\item
  Vorteile guter Usability:

  \begin{itemize}
  \item
    gesteigerte Produktivität
  \item
    Glückliche Benutzer
  \item
    Weniger Kosten (Zeit, Geld, Gesundheit) (?)
  \end{itemize}
\item
  Es gibt Theorien, Prinzipien und Richtlinien (abstrakt nach konkret):
\end{itemize}

\subsection{Theorien}

\begin{itemize}
\item
  Kognition: GOMS, ACT-R
\item
  Sinne: Sehen, Hören, Fühlen
\item
  Bewegung: Fitts' Law
\end{itemize}

\subsubsection{Fitts' Law}

\begin{itemize}
\item
  Modell für die motorische Bewegung
\item
  Besonders für schnelles Zielen
\item
  Beschreibendes und vorhersehendes Modell
\item
  Die Schwierigkeit einer Bewegung ist abhängig von der
  \emph{zurückzulegenden Distanz} und der \emph{Größe des Ziels}
\item
  Kanten und Ecken sind am Besten zu erreichen
\end{itemize}

\subsection{Prinzipien}

\begin{itemize}
\item
  Shneiderman's 8 Golden Rules of Interface Design
\item
  Niesen's 10 Heuristics for User Interface
\item
  Tognazzini's First (16) Principles of Interface Design
\end{itemize}

\subsubsection{8 Goldene Regeln für Interface Design}

\begin{enumerate}[1.]
\item
  Konsistenz: Reihenfolge von Handlungen, Begriffe, Design
\item
  Universale Usability: Menschen sind unterschiedlich
\item
  Informative Rückmeldung: für \emph{jede} Handlung muss es Feedback
  geben
\item
  Abschließen von Dialogen: Nach Beendigung einer Aufgabe muss es
  abschließendes Feeback geben
\item
  Fehler verhindern: z.B. falsche Eingaben
\item
  Einfaches Rückgängig machen: gibt dem Benutzer Sicherheit
\item
  Benutzerkontrolle: Der Benutzer sollte immer die Kontrolle haben
\item
  Kurzzeitgedächtnis entlasten: es können nur etwa 7 $(\pm2)$
  ``Datenpakete'' gemerkt werden
\end{enumerate}

\subsection{Richtlinien}

\begin{itemize}
\item
  Finden sich z.B. oft in Betriebssystemen
\end{itemize}

\begin{enumerate}[1.]
\item
  Navigation: Linktext sollte immer aussagekräftig sein, Überschriften
  eindeutig und beschreibend
\item
  Organisation der Anzeige: Datenformate sollten einheitlich und bekannt
  sein, Eingabe sollte Anzeige entsprechen, Ausgabe sollte editierbar
  sein
\item
  Aufmerksamkeit erlangen:

  \begin{itemize}
  \item
    2 Stufen Instensität (Fettdruck)
  \item
    Unterstreichungen oder Pfeile
  \item
    Bis zu 4 Schriftgrößen
  \item
    Bis zu 3 Schriftarten
  \item
    Kein Blinken
  \item
    Bis zu 4 Farben
  \item
    Sanfte Töne = gut / Harte Töne = Fehler
  \end{itemize}
\end{enumerate}

\subsection{Standards}

\subsubsection{ISO 9241}

Dialogprinzipien nach ISO 9241-110:

\begin{enumerate}[1.]
\item
  Angemessenheit: Der Dialog sollte den Nutzer unterstützen
\item
  Selbsterklärung: entweder sofort verständlich oder auf Anfrage mit
  Hilfe versehen
\item
  Kontrollierbarkeit: Der Benutzer kontrolliert, nicht der Computer
\item
  Übereinstimmung mit Erwartungen
\item
  Fehlertoleranz: Fehler sollen mehr oder weniger automatisch behoben
  werden
\item
  Möglichkeit der Individualisierung
\item
  Lernmöglichkeiten
\end{enumerate}

\subsection{User Experience vs.~Usability}

User Experience = Usability + Motivation + Emotionen + Werte

\section{Prototyping}

\begin{itemize}
\item
  Warum?

  \begin{itemize}
  \item
    Prototypen eignen sich für Nutzerstudien, den Nutzer wissen nicht,
    was sie wollen, sehr wohl aber was sie nicht wollen.
  \item
    Man kann Fragen beantworten (Funktioniert das Konzept?)
  \item
    Alternativen vergleichen
  \end{itemize}
\item
  Wann?

  \begin{itemize}
  \item
    Je frühe, umso besser
  \end{itemize}
\item
  Was?

  \begin{itemize}
  \item
    Alles
  \end{itemize}
\item
  Ansätze

  \begin{itemize}
  \item
    Wegwerfprototypen (``rapid prototype'')
  \item
    Evolutionärerprototyp (wird weiterentwickelt)
  \item
    Inkrementeller Prototyp (ein Teil des Ganzen, wird später eingefügt)
  \item
    Horizontal (viele Features, wenig Funktionalität) $\leftrightarrow$
    Vertikal (ein Feature, volle Funktionalität)
  \item
    Li-Fi-Prototype (früh, billig, oberflächlich) $\leftrightarrow$
    Hi-Fi-Prototype (viele Details)
  \end{itemize}
\item
  Techniken

  \begin{itemize}
  \item
    Storyboarding
  \item
    Paper-Prototype
  \item
    Click-Prototype (GUI, z.B. Pidoco)
  \item
    Wizard-Of-Oz-Prototype (Mensch ersetzt Funktionalität)
  \end{itemize}
\end{itemize}

\section{Usability Evaluation 1 --- Testing with Users}

\section{Usability Evaluation 2 --- Analytical and Expert Methods}
